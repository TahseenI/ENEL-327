\documentclass{beamer}
\usepackage[utf8]{inputenc}

\usepackage{amsmath, mathtools}
\usepackage{amssymb}
\usepackage{mathrsfs}
\usepackage{concmath}
\usepackage{extarrows}
\usepackage{tikz}
\usetheme{Copenhagen}
\usecolortheme{beaver}
\usetikzlibrary{patterns, backgrounds, positioning}
\newcommand{\dt}[0]{\mathrm{d}t}
\renewcommand{\Re}[0]{\mathrm{Re}} 
\renewcommand{\Im}[0]{\mathrm{Im}} 

\title{Region of Convergence and Common Z-Transforms}
%\author{Tahseen Intesar}
\date{April 18, 2020}
\begin{document}
\maketitle

\begin{frame}{Recall from Last Slide-set}

    \begin{exampleblock}{Definition of a Z-transform}
      \begin{equation*}
        \mathnormal{{X(z)} = \mathcal{Z}\{x[n]\} = \sum_{n=-\infty}^{\infty} x[n] z^{-n}}
      \end{equation*}
      ...where \emph{z}, the independent variable, is a complex variable.
    \end{exampleblock}
    
    \begin{itemize}
        \item The Z-transform is the discrete-time equivalent of the Laplace transform
        
        \item The Discrete-Time Fourier Transform (DTFT) of a signal x[n] is equal to its z-transform evaluated at each point on the unit circle of the z-plane described by the trajectory $z=e^{j\Omega}$.
    \end{itemize}
    
\end{frame}

\begin{frame}{Region of Convergence (ROC)}
    \begin{itemize}
        \item The definition of the Z-Transform contains a power series. The negative exponent on z allows for the sum of the terms to converge
    \end{itemize}

    %The following should be fixed for Z-transform
    \begin{exampleblock}{Condition for a Region of Convergence}
      \begin{equation*}
        \mathnormal{\sum_{n=-\infty}^{\infty} \left|x[n] z^{-n}\right| < \infty}
      \end{equation*}
    \end{exampleblock}
    
    \begin{itemize}
        \item X(z) cannot help us determine x(t) alone! 
    \end{itemize}
    
     \begin{alertblock}{Definition of the Region of Convergence}
      The ROC is a collection of all points on the z-plane in which the sum converges.
    \end{alertblock}
    
\end{frame}

\begin{frame}{Example: Simple Z-Transform of a Signal}
    \begin{center}
    $ x[n] = \{3.7, 1.3, -1.5, 3.4, 5.2\} \longrightarrow x[n=0] = 3.7 $
    \end{center}
    The Z-Transform is :

    \begin{align*}
    {X(z)} &= \sum_{n=-\infty}^{\infty} x[n] z^{-n} \\
           &= x[0]z^{0} + x[1]z^{-1} + x[2]z^{-2} + x[3]z^{-3} + x[4]z^{-4}\\
           &= 3.7 + 1.3z^{-1} - 1.5z^{-2} + 3.4z^{-3} + 5.2z^{-4}\\
    \end{align*}
    
    ... which yields a polynomial with a degree of -1.
\end{frame}
\begin{frame}{Example: Simple Z-Transform of a Signal}
    As you may have noticed, the coefficients on each term contain the information of the time-domain samples.
    \newline \newline
    Taking $z_1 = 1 + 2j$, the value of the transform is:
    \begin{align*}
    {X(1+2j)} &= \sum_{n=-\infty}^{\infty} x[n] z^{-n} \\
              &= 3.7 + 1.3(1+2j)^{-1} - 1.5(1+2j)^{-2} + 3.4(1+2j)^{-3}\\ &+ 5.2(1+2j)^{-4}\\
              &= 3.7 + \frac{1.3}{(1+2j)} - \frac{1.5}{(1+2j)^2} + \frac{3.4}{(1+2j)^3} + \frac{5.2}{(1+2j)^4} \\
              &= 3.7826 - j0.0259
    \end{align*}
    Does this work for all values of $z$? %no, 0,0 is the exception
\end{frame}

\begin{frame}{Unit Impulse}
    \begin{center}
    $ x[n] = \delta[n] =  
    \begin{cases}
        1 & n = 0\\
        0 & n \neq 0\\
    \end{cases} $
    \end{center}
    
    Since our only non-zero sample is at n = 0, the Z-Transform becomes:
    \begin{align*}
    {X(z)} &= \sum_{n=-\infty}^{\infty} x[n] z^{-n} = x[0]z^{0} = 1
    \end{align*}
    The impulse signal yields a constant value (no exponent). Because of this, we can say that the sum converges at every point on the z-plane. In other words, 
    \textbf{the ROC is the entire z-plane}.
\end{frame}

\begin{frame}{Shifted Unit Impulse}
    \begin{center}
    $x[n-k] = \delta[n-k] =  
    \begin{cases}
        1 & n = k\\
        0 & n \neq k\\
    \end{cases}
    where \ k \neq 0, \in \mathbb{Z}$
    \end{center}
    Using the same procedure as before, we get:
    \begin{align*}
    {X(z)} &= \sum_{n=-\infty}^{\infty} x[n] z^{-n} = x[k]z^{-k} = \boxed{z^{-k}}
    \end{align*}
    Looking at the cases for k:
    $
    \begin{cases}
        k < 0: & transform \ does \ not \ converge \ at \ infinite \ radius\\
        k > 0: & transform \ does \ not \ converge \ at \ origin\  (0 + 0j)\\
    \end{cases}
    $
    \linebreak
    Otherwise, the transform converges at \textbf{all points on the z-plane.}
\end{frame}

\begin{frame}
    \begin{center}{Here are the functions we discussed and their corresponding ROCs}
        \begin{table}
            \begin{tabular}{| c | c |}
            \hline
                \textbf{Function} & \textbf{Region of Convergence}\\
            \hline
                 \delta[n] & entire z-plane \\
            \hline
                 \delta[n-k] & entire z-plane except for z = 0 and z = \infty \\
            \hline
                  x[n] & entire z-plane except for z = 0 \\
            \hline
            \end{tabular}
        \end{table}
    \end{center}
    
    \textit{Note: $x[n]$ is a signal with given values. (refer to the first example)}
    
    Next, we will look at the ROC for causal and non-causal signals.
\end{frame}

\begin{frame}{Nature of ROC}
    \begin{figure}[ht]
        \begin{minipage}[l]{.32\linewidth} %figure 1
          \begin{tikzpicture}[scale = 0.4]
            \draw[<->,thick] (-4,0)--(4,0);
            \draw[<->,thick] (0,-4)--(0,4) node[above]{\tiny{$\Im(z)$}}; 
            \draw[red, ->, -stealth] (0,0)--(2,2.2);
            \node[align = center] at (2.5,2.5) {\tiny{$r_1$}};
            \path [draw=none, pattern = north east lines, pattern color = gray!50,even odd rule] (0,0) circle (3) (0,0);
            
            \node[red, align = center] (ROC) at (-3.75,2.55) {\tiny ROC};
            \draw[->] (ROC) (-2.8,2.25) -- (-1.75, 1.75);
            \draw[red, dashed] (0,0) circle (3); % let the basic radius = 3
            \end{tikzpicture}
        \end{minipage}
        \begin{minipage}[r]{.33\linewidth} %figure 2
          \begin{tikzpicture}[scale = 0.4]
        % You can change this and put the point anywhere on the z-plane
            \coordinate (POINT) at (2.1,2.1);
            \draw[<->,thick] (-4,0)--(4,0);
            \draw[<->,thick] (0,-4)--(0,4) node[above]{\tiny{$\Im(z)$}}; 
            \path[draw=none, pattern = north east lines, pattern color = gray!50,even odd rule] (0,0) circle (3) (-4,-4) rectangle (4,4);
            \draw[red, dashed] (0,0) circle (3);
        % If you want a dot at z = r
        %\fill[gray] (POINT) circle (0.1);
            \draw[red, ->, -stealth] (0,0)--(POINT);
            \node[align = right] at (2, 1.1) {\tiny{$r_2$}};
                \end{tikzpicture}
            \end{minipage}
        \begin{minipage}[r]{.32\linewidth} %figure 3
            \begin{tikzpicture}[scale = 0.4]
            \draw[<->,thick] (-4,0)--(4,0) node[right]{\tiny{$\Re(z)$}};
            \draw[<->,thick] (0,-4)--(0,4) node[above]{\tiny{$\Im(z)$}}; 
            \path [draw=none, pattern = north east lines, pattern color = gray!50,even odd rule] (0,0) circle (2) circle (4);
            \draw[red, dashed] (0,0) circle (2); 
            \draw[red, dashed] (0,0) circle (4);
            
            \draw[red, ->, -stealth] (0,0)--(1,-1.6); %r1
            \draw[red, ->, -stealth] (0,0)--(2.25,3.3); %r2
            
            \node[align = right] at (1, -0.5) {\tiny{$r_1$}};
            \node[align = right] at (2.8,3.5) {\tiny{$r_2$}};
            
            \end{tikzpicture}
         \end{minipage}
    \end{figure}
    
    Possible ROC Shape for a signal :
    \begin{description} [align=right, labelwidth = 1.75cm]
        \item [$r < r_1$] : Inside a circle
        \item [$r > r_2$] : Outside a circle
        \item [$r_1 < r < r_2$] : Between 2 circles 
    \end{description}
    
\end{frame}

\begin{frame}{Unit Step Function}
    \begin{center}
        $ x[n] = u[n] =  
        \begin{cases}
            1 & n \geq 0\\
            0 & n < 0\\
        \end{cases} $
    \end{center}
    Again, we apply the z-transform to the function, but now the lower limit for the sum is 0 since $u[n] = 0$ for all  $n < 0$ :
    \begin{align*}
    {X(z)} &= \sum_{n=0}^{\infty} u[n] z^{-n}
    \end{align*}
    
    Since $u[n] = 1$ for all $n \geq 0$, we can remove the u[n] term from the summation.
    \begin{align*}
    {X(z)} &= \sum_{n=0}^{\infty} z^{-n}
    \end{align*}
    But what do we do from here?
\end{frame}
\begin{frame}{Unit Step Function}
    The formula for an infinite geometric series is :
    \begin{align*}
    {X(z)} &= \sum_{n=0}^{\infty} z^{-n} \\
           &= \frac{1}{1-z^{-1}} = \boxed{\frac{z}{z-1}}
    \end{align*}
    
    ...which converges for : $|z^{-1}| < 1$ and $|z| > 1$
    \linebreak \linebreak
    The ROC of a Unit Step Function is the collection of points outside of a \textbf{circle of radius 1.}
    
    % NEED A FIGURE FOR THE UNIT STEP ROC (PAGE 734 IN TEXTBOOK)
    % the figure can be placed on this slide or the next
    % the unit circle itself is not part of the ROC since the transform does not converge for the values on the circle
    
\end{frame}
\begin{frame}{Unit Step Function}
    
    \begin{figure}
        \begin{tikzpicture}[scale = 0.7]
        % You can change this and put the point anywhere on the z-plane
            \coordinate (POINT) at (2.1,2.1);
            \coordinate (LABEL) at (-3, 3);
            
            \draw[<->,thick] (-4,0)--(4,0) node[right]{$\Re(z)$};
            \draw[<->,thick] (0,-4)--(0,4) node[above]{$\Im(z)$};
            
            \path[draw=none, pattern = north east lines, pattern color = gray!80,even odd rule] (0,0) circle (3) (-4,-4) rectangle (4,4);
            \draw[red, dashed] (0,0) circle (3);
            
            % If you want a dot at z = r
            %\fill[gray] (POINT) circle (0.1);
            
            \draw[red, ->, -stealth] (0,0)--(POINT);
            \node[align = right] at (2, 1.1) {\small{$r = 1$}};
            \node[red, -stealth] at (LABEL) {\large{ROC}};
            \node[red, -stealth] at (1.1, -1.1) {\large{$|z| < 1$}};
            \node[red, -stealth] at (3, -3) {\large{$|z| > 1$}};
            
            
        \end{tikzpicture}
    \end{figure}
    
\end{frame}

\begin{frame}{Causal  Signal}
    $x[n] = a^{n}u[n] =  
    \begin{cases}
        a^{n} & n \geq 0\\
        0 & n < 0\\
    \end{cases}
    $ Where $a$ is any real or complex value.
    
    \begin{align*}
    {X(z)} &= \sum_{n=-\infty}^{\infty} a^{n} z^{-n} = \sum_{n=-\infty}^{\infty} (az^{-1})^{n}
    \end{align*}
    
    And similarly, we can find that :
    
    \begin{align*}
    {X(z)} &= \sum_{n=0}^{\infty} az^{-n} \\ 
           &= \frac{1}{1-az^{-1}} = \boxed{\frac{z}{z-a}}
    \end{align*}
    
    ...which converges for : $|az^{-1}| < 1$ , $\frac{|a|}{|z|} > 1$ , $|z| > |a|$
    
    %In a new slide, need plots for |a| < 1 and |a| > 1
    
\end{frame}
\begin{frame}{Causal Signal}
    \begin{figure}[ht]
        \begin{minipage}[l]{.45\linewidth}
        
        \begin{tikzpicture}[scale = 0.45]
        % You can change this and put the point anywhere on the z-plane
            \coordinate (POINT) at (2.1,2.1);
            \coordinate (LABEL) at (-3, 3);
            
            \draw[<->,thick] (-4,0)--(4,0) node[right]{$\Re(z)$};
            \draw[<->,thick] (0,-4)--(0,4) node[above]{$\Im(z)$};
            
            \path[draw=none, pattern = north east lines, pattern color = gray!50,even odd rule] (0,0) circle (3) (-4,-4) rectangle (4,4);
            \draw[red, dashed] (0,0) circle (3);
            \draw[red, dashed] (0,0) circle (2);
            \draw[red, ->, -stealth] (0,0)--(-1,1.6);
            \draw[red, ->, -stealth] (0,0)--(POINT);
            
            % If you want a dot at z = r
            %\fill[gray] (POINT) circle (0.1);
            
            \node[align = right] at (-1.2,0.5) {\tiny{$r = 1$}};
            \node[align = right] at (3, 2.2) {\tiny{$r = a$}};
            \node[red, -stealth] at (LABEL) {\small{ROC}};
            \node[red, -stealth] at (0.2, 2.4) {\tiny{$|z| < |a|$}};
            \node[red, -stealth] at (2.7, -3.1) {\tiny{$|z| > |a|$}};
            
        \end{tikzpicture}
        
        \end{minipage}
        \begin{minipage}[r]{.45\linewidth} 
        
        \begin{tikzpicture}[scale = 0.45]
        % You can change this and put the point anywhere on the z-plane
            \coordinate (POINT) at (2.1,2.1);
            \coordinate (LABEL) at (-3, 3);
            
            \draw[<->,thick] (-4,0)--(4,0) node[right]{$\Re(z)$};
            \draw[<->,thick] (0,-4)--(0,4) node[above]{$\Im(z)$};
            
            \path[draw=none, pattern = north east lines, pattern color = gray!50,even odd rule] (0,0) circle (2) (-4,-4) rectangle (4,4);
            \draw[red, dashed] (0,0) circle (3);
            \draw[red, dashed] (0,0) circle (2);
            \draw[red, ->, -stealth] (0,0)--(-1,1.6);
            \draw[red, ->, -stealth] (0,0)--(POINT);
            
        % If you want a dot at z = r
        %\fill[gray] (POINT) circle (0.1);
            
            \node[align = right] at (-1.2, 0.5) {\tiny{$r = a$}};
            \node[align = right] at (3, 2.2) {\tiny{$r = 1$}};
            \node[red, -stealth] at (LABEL) {\small{ROC}};
            \node[red, -stealth] at (0.2, -1.2) {\tiny{$|z| < |a|$}};
            \node[red, -stealth] at (2.7, -3.1) {\tiny{$|z| > |a|$}};
            
        \end{tikzpicture}
        
        \end{minipage}
    \end{figure}
    
    The left figure shows the ROC for a causal signal with $|a| > 1$, and the figure on the right shows the same for $|a| < 1$
\end{frame}

\begin{frame}{Non Causal Signal}
    $ x[n] = -a^{n}u[-n-1] =  
        \begin{cases}
            -a^{n} & n < 0\\
            0 & n \geq 0\\
        \end{cases}
    $ 
    \linebreak
    Where $a$ is any real or complex value. \linebreak
    Note that: 
    $ u[-n-1] =  
        \begin{cases}
            1 & n < 0\\
            0 & n \geq 0\\
        \end{cases}
    $ 
    \linebreak
    If we change the upper limit of the summation to $n=-1$, the $u[-n-1]$ term becomes obsolete.
    \begin{align*}
    {X(z)} &= \sum_{n=-\infty}^{\infty} a^{n}u[-n-1]z^{-n} \\
           &= - \sum_{n=-\infty}^{-1} a^{n}z^{-n} \\
           &= - \sum_{n=-\infty}^{-1} (az^{-1})^{n} \\
    \end{align*}
    
\end{frame}
\begin{frame}{Non Causal Signal}
    If we let m = -n, our expression becomes : 
    \begin{align*}
    {X(z)} &= - \sum_{m=-1}^{\infty} (a^{-1} z)^{m}
    \end{align*}
    We will now apply another variable change of k = m - 1, so that we get k = 0 in our lower summation limit : 
    \begin{align*}
    {X(z)} &= - \sum_{k = 0}^{\infty} (a^{-1} z)^{k+1} \\
           &= - a^{-1} z \sum_{k = 0}^{\infty} (a^{-1} z)^{k} \\
           &= - a^{-1} z (\frac{1}{1 - a^{-1}z}) & \boxed{= \frac{z}{z-a}}
    \end{align*}
    
    ...which converges for : $|a^{-1}z| < 1$ , $\frac{|z|}{|a|} < 1$ , $|z| < |a|$
    
     %In a new slide, need plots for |a| < 1 and |a| > 1
    
\end{frame}
\begin{frame}{Non Causal Signal}

    \begin{minipage}[l]{.45\linewidth}
          \begin{tikzpicture}[scale = 0.45]
            \draw[<->,thick] (-4,0)--(4,0) node[right]{$\Re(z)$}; 
            \draw[<->,thick] (0,-4)--(0,4) node[above]{$\Im(z)$}; 
            \draw[red, ->, -stealth] (0,0)--(2,2.2);
            \path [draw=none, pattern = north east lines, pattern color = gray!50,even odd rule] (0,0) circle (2.6) (0,0);
            
            \draw[red, dashed] (0,0) circle (3);
            \draw[red, dashed] (0,0) circle (2.6);
            \draw[red, ->, -stealth] (0,0) -- (-1.5, -2);
            \draw[red, ->] (ROC) (-3,2.5) -- (-2.5, 1.9);
            
            \node[red, align = center] (ROC) at (-4,3) {\tiny Unit Circle};
            \node[align = center] at (2.9,2.5) {\tiny{$r = 1$}};
            \node[align = center] at (-2,-2.4) {\tiny{$r = |a|$}};
            \node[align = center] at (3,-3) {\tiny{$|z| > |a|$}};
            \node[align = center] at (1,-1) {\tiny{$|z| < |a|$}};
            
        \end{tikzpicture}
    \end{minipage}
    \begin{minipage}[l]{.45\linewidth}
          \begin{tikzpicture}[scale = 0.45]
            \draw[<->,thick] (-4,0)--(4,0) node[right]{$\Re(z)$}; 
            \draw[<->,thick] (0,-4)--(0,4) node[above]{$\Im(z)$}; 
        
            \path [draw=none, pattern = north east lines, pattern color = gray!50,even odd rule] (0,0) circle (3.5) (0,0);
            
            \draw[red, dashed] (0,0) circle (3.5);
            \draw[red, dashed] (0,0) circle (2.5);
            \draw[red, ->, -stealth] (0,0)--(2.5,2.5);
            \draw[red, ->, -stealth] (0,0) -- (-1.5, -2);
            \draw[red, ->] (ROC) (-3,2.5) -- (-1.5, 2);
            
            \node[red, align = center] (ROC) at (-4,3) {\tiny Unit Circle};
            \node[align = center] at (3.4,2.5) {\tiny{$r = |a|$}};
            \node[align = center] at (-1.8,-2.4) {\tiny{$r = 1$}};
            \node[align = center] at (3.4,-3) {\tiny{$|z| > |a|$}};
            \node[align = center] at (1,-1) {\tiny{$|z| < |a|$}};
            
        \end{tikzpicture}
    \end{minipage}
    
    The left figure shows the ROC for a causal signal with $|a| > 1$, and the figure on the right shows the same for $|a| < 1$
\end{frame}

\begin{frame}{Zeros and Poles}
    Let's say we are given a Z-transform of : \\
    \begin{center}
        $X(z) = \frac{z}{z-a}$ \\
    \end{center}
    
    \begin{itemize}
        \item We don't know if the original signal is causal or non-causal, since we are not given the conditions for $z$ and $a$. 
    \end{itemize}
    
    What we do know is that the transform is generally expressed in the form : \\
    \begin{center}
        $X(z) = K \frac{B(z)}{A(z)}$ 
    \end{center}
    ...where $B(z)$ and $A(z)$ are polynomials, K is the gain factor.
\end{frame}
\begin{frame}{Zeros and Poles}
Using the factored form for the polynomials, we get :
    \begin{align*}
         X(z) = K \frac{(z - z_1)(z - z_2) ..., (z - z_M)}{(z - p_1)(z - p_2) ..., (z - p_N)}
    \end{align*}
    ...where M and N are the orders of the polynomials on the numerator and denominator respectively. \linebreak
    \begin{itemize}
        \item The roots of the polynomials in the numerator are \textbf{zeros}.
        \item The roots of the polynomials in the denominator are \textbf{poles}.
    \end{itemize}
    
    Why are they important?
\end{frame}

\begin{frame}{Summary of Region of Convergence}
    \begin{itemize}
        \item ROC is circular shaped: inside of a circle, outside of a circle or between two circles
        \item ROC cannot contain poles
        \item For a causal signal, the ROC is $|z| > r_1$
        \item For a non causal signal, the ROC is $|z| < r_2$
        \item For a signal that is neither causal or non causal, the ROC is $r_1 < |z| < r_2$
    \end{itemize}
\end{frame}

\begin{frame}{Properties of Z-Transforms}
    \begin{enumerate}
        \item Linearity
        \item Time Shifting
        \item Time Reversal
        \item Differentiation
        \item Convolution
    \end{enumerate}
\end{frame}

\begin{frame}{Linearity}
    For any two signals with their respective transforms :
    \begin{align*}
        x_1[n] \xlongleftrightarrow{ \mathcal{Z} } X_1(z) \quad
        x_2[n] \xlongleftrightarrow{ \mathcal{Z} } X_2(z)
    \end{align*}
    And two constants: $\alpha_1$ , $\alpha_2$ , we can find the Z-transform:
    \begin{align*}
        \mathcal{Z}\{\alpha_1 x_1[n] + \alpha_2 x_2[n]\} &= \sum_{n=-\infty}^{\infty} (\alpha_1 x_1[n] + \alpha_2 x_2[n])z^{-n} \\
            &= \sum_{n=-\infty}^{\infty} \alpha_1 x_1[n]z^{-n} +  \sum_{n=-\infty}^{\infty} \alpha_2 x_2[n]z^{-n} \\
            &= \alpha_1 \cdot \sum_{n=-\infty}^{\infty} x_1[n]z^{-n} + \alpha_2 \cdot \sum_{n=-\infty}^{\infty} x_2[n]z^{-n} \\
            \Aboxed{&= \alpha_1 \cdot \mathcal{Z}\{x_1[n]\} + \alpha_2 \cdot \mathcal{Z} \{x_2[n]\}} \\
    \end{align*} 

\end{frame}

\begin{frame}{Time Shifting}
    Given the transform pair : 
    \begin{align*}
        x[n] \xlongleftrightarrow{ \mathcal{Z} } X(z)
    \end{align*}
    For a time shifted signal x[n-k] : 
    \[ \mathcal{Z}\{ x[n-k] \} = \sum_{n=-\infty}^{\infty}  x[n-k] \cdot z^{-n} \]
    If we define $m = n - k$ :
    \begin{align*}
        &= \sum_{m=-\infty}^{\infty}  x[m] \cdot z^{-(m+k)} \\
        &= z^{-k} \sum_{m=-\infty}^{\infty}  x[m] \cdot z^{-m} \\
        \Aboxed{&= z^{-k} \cdot X(z)}
    \end{align*}
\end{frame}

\begin{frame}{Time Reversal}
For a reversed time signal x[-n] :

    \[ \mathcal{Z}\{ x[-n] \} = \sum_{n=-\infty}^{\infty}  x[-n] \cdot z^{-n} \]
    If we define $m = -n$ :
    
    \begin{align*}
        \mathcal{Z}\{ x[-n] \} &= \sum_{m=+\infty}^{-\infty}  x[m] \cdot z^{m} \\
    \end{align*}
     We can switch the limits of the summation, and factor the 'm' out of the exponent to get : 
    
    \begin{align*}
       \mathcal{Z}\{ x[-n] \} &= \sum_{m=-\infty}^{+\infty}  x[m] \cdot (z^{-1})^{m} \Aboxed{&= z^{-k} \cdot X(z^{-1})}
    \end{align*}{}
    
\end{frame}

\begin{frame}{Differentiation}
    Starting with the definition :
    \begin{align*}
        X(z) &= \sum_{n=-\infty}^{\infty} x[n] \cdot z^{-n}
    \end{align*}
    We can differentiate both sides with respect to z :
    \begin{align*}
        \frac{d}{dz}[X(z)] &= \frac{d}{dz}[\sum_{n=-\infty}^{\infty} x[n] \cdot z^{-n}] \\
    \end{align*}
    Since summation and differentiation are both linear, the order can be reversed.
    \begin{align*}
        \frac{d}{dz}[X(z)] &= \sum_{n=-\infty}^{\infty} \frac{d}{dz} [x[n] \cdot z^{-n}] \\
    \end{align*}{}
\end{frame}
\begin{frame}{Differentiation}
    Now, by differentiating the term inside the sum :
    \begin{align*}
        &= \sum_{n=-\infty}^{\infty} -nx[n]z^{-n-1} \\
        &= -z^{-1} \sum_{n=-\infty}^{\infty} -nx[n]z^{-n} \\
        \frac{d}{dz}[X(z)] &= \frac{1}{(-z)} \sum_{n=-\infty}^{\infty} -nx[n]z^{-n}\\
        \Aboxed{(-z) \cdot \frac{d}{dz}[X(z)] &= \mathcal{Z}\{ nx[n] \}}
    \end{align*}
        
\end{frame}

\begin{frame}{Convolution}
    The convolution of two discrete time signals is given by :
    \begin{align*}
        x_1[n] \ast x_2[n] = \sum_{k=-\infty}^{\infty} x_1[k]\cdot x_2[n-k] \\
    \end{align*}
    By applying the Z-transform :
    \begin{align*}
        \mathcal{Z}\{ x_1[n] \ast x_2[n] \} &= \mathcal{Z}\{\sum_{k=-\infty}^{\infty} x_1[k] \cdot x_2[n-k] \} \\
        &= \sum_{n=-\infty}^{\infty} (\sum_{k=-\infty}^{\infty} x_1[k] \cdot x_2[n-k]) z^{-n}
    \end{align*}
\end{frame}
\begin{frame}{Convolution}
    Changing the order of summation and multiplication gives us :
    \begin{align*}
        &= \sum_{k=-\infty}^{\infty} x_1[k] \sum_{n=-\infty}^{\infty} x_2[n-k] z^{-n} \\
    \end{align*}
    The inner summation term represents the Z-transform of $x_2[n-k]$, which means :
    \begin{align*}
        \mathcal{Z}\{ x_1[n] \ast x_2[n] \} &= \sum_{k=-\infty}^{\infty} x_1[k] z^{-k} \cdot X_2(z)\\
        &= X_2(z) \cdot \sum_{k=-\infty}^{\infty} x_1[k] z^{-k}
        \Aboxed{&= X_1(z) \cdot X_2(z)}
    \end{align*}
\end{frame}



\end{document}
